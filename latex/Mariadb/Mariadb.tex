\documentclass[a4paper, 11pt, titlepage]{article} 

\usepackage{hyperref}
\usepackage[utf8]{inputenc}
\usepackage[spanish]{babel}
\usepackage{mathtools}
\usepackage{graphicx}

% CODE C
\usepackage{amssymb}
\usepackage{listings}
\usepackage{verbatim}
\usepackage{xcolor}
\definecolor{textblue}{rgb}{.2,.2,.7}
\definecolor{textred}{rgb}{0.54,0,0}
\definecolor{textgreen}{rgb}{0,0.43,0}
\lstset{language=C, 
numbers=left, 
numberstyle=\tiny, 
stepnumber=1,
numbersep=5pt, 
tabsize=4,
basicstyle=\ttfamily,
keywordstyle=\color{textblue},
commentstyle=\color{textred},   
stringstyle=\color{textgreen},
frame=none,                    
columns=fullflexible,
keepspaces=true,
xleftmargin=\parindent,
showstringspaces=false}

\title{Mariadb}
\author{Alberto Fraile}
\date{2020}

\begin{document}

\maketitle
\tableofcontents
\newpage

\section*{Acceso a Mariadb}
\addcontentsline{toc}{section}{Acceso a Mariadb}

\begin{verbatim}
sudo mariadb -u root -p 
\end{verbatim}

\section*{Creación y acceso a la base de datos} 
\addcontentsline{toc}{section}{Creación y estructura de la base de datos}

\begin{verbatim}
CREATE DATABASE Ejemplo;
\end{verbatim}
<<Crear nueva base de datos>> 
\begin{verbatim}
SHOW DATABASES; 
\end{verbatim}
<<Visualizar las bases de datos existentes>> 
\begin{verbatim}
USE Ejemplo; 
\end{verbatim}
<<Acceder a la base de datos seleccionada>> \\

\section*{Creación y modificación de tablas}
\addcontentsline{toc}{section}{Creación y modificación de tablas}

\begin{verbatim}
CREATE TABLE Ejemplo ( 
ID int(10) not null, auto_increment,
Nombre varchar(20) not null,
Apellido varchar(20) not null,
PRIMARY KEY (ID) );
\end{verbatim}
<<Creación de tabla en la base de datos>> 
\begin{verbatim}
SHOW TABLES Ejemplo; 
\end{verbatim}
<<Visualizar tablas de la base de datos>> 
\begin{verbatim}
DESCRIBE Ejemplo;
\end{verbatim}
<<Muestra los campos contenidos en una tabla>> 
\begin{verbatim}
RENAME TABLE Ejemplo \textbf{TO} Personas;
\end{verbatim}
<<Renombrar tabla>>
\begin{verbatim}
ALTER TABLE Ejemplo ADD id int(10), not null, auto_increment;
\end{verbatim}
<<Añadir nuevo campo a la tabla>> 
\begin{verbatim}
ALTER TABLE Ejemplo CHANGE id ID;
\end{verbatim}
<<Cambiar nombre de la tabla>> \\
\begin{verbatim}
ALTER TABLE Ejemplo CHANGE Nombre Nombre VARCHAR(20);
\end{verbatim}
<<Cambiar tipo de dato de la tabla>> 
\begin{verbatim}
DROP TABLE Personas;
\end{verbatim}
<<Eliminar tabla>> 
\begin{verbatim}
DELETE FROM Ejemplo WHERE Edad =10;
\end{verbatim}
<<Eliminar registros de una tabla segun condiciones>> 
\begin{verbatim}
UPDATE Ejemplo SET Nombre, Edad WHERE ID =1;
\end{verbatim}
<<Actualizar registros de la tabla>> 
\begin{verbatim}
BACKUP DATABASE Ejemplo TO DISK = Ubicacion;
\end{verbatim}
<<Backup de la base de datos>> 
\begin{verbatim}
BACKUP DATABASE Ejemplo TO DISK = Ubicacion WITH DIFFERENTIAL;
\end{verbatim}
<<Backup de las partes cambiadas desde la ultima vez>> 
\begin{verbatim}
INSERT INTO Ejemplo
(ID, Nombre, Apellido, Edad)
VALUES
(1, Alberto, Hernandez, 40),
(2, Javier, Sevilla, 45),
(3, Alfredo, Martinez, 65);
\end{verbatim}
<<Insertar múltiples registros en una tabla>>

\section*{Consultas}
\addcontentsline{toc}{section}{Consultas}
\begin{verbatim}
SELECT * FROM Ejemplo;
\end{verbatim}
<<Consultar todo el contenido de la tabla>> 
\begin{verbatim}
SELECT ID, Nombre FROM Ejemplo; 
\end{verbatim}
<<Consultar el contenido de la tabla por columnas>> 
\begin{verbatim}
SELECT ID, Nombre FROM Ejemplo WHERE Edad = 40;
\end{verbatim}
<<Consultar el contenido por columnas de la tabla con condiciones>> 
\begin{verbatim}
SELECT Nombre FROM Ejemplo WHERE Edad = 40 AND Nombre = Alberto; 
\end{verbatim}
<<Conultar el contenido con mas de una condicion>> 
\begin{verbatim}
SELECT * FROM Ejemplo WHERE Nombre IS NOT NULL / IS NULL;
\end{verbatim}
<<Comprobar si hay resistros vacios en los distintos campos de la tabla>> 
\begin{verbatim}
 SELECT * FROM Ejemplo ORDER BY Estatura;
\end{verbatim}
 <<Consultar el contenido de la tabla ordenado por parametros>> 
\begin{verbatim}
 SELECT * FROM Ejemplo ORDER BY Estatura DESC;
\end{verbatim}
 <<Consultar el contenido de la tabla por orden descendente>> 
\begin{verbatim}
 SELECT * FROM Ejemplo ORDER BY Estatura ASC;
\end{verbatim}
 <<Consultar el contenido de la tabla por orden ascendente>> 
 \begin{verbatim}
 SELECT * FROM Ejemplo ORDER BY Estatura DESC, Peso ASC; \end{verbatim}
 <<Consultar el contenido de la tabla con distintos parámetros y órdenes>> 
 \begin{verbatim}
 SELECT TOP 4 FROM Ejemplo;
\end{verbatim}
 <<Devuelve un numero determinado de registros>> 
\begin{verbatim} 
SELECT MAX(Ejemplares) FROM Ejemplo WHERE ID =1;
\end{verbatim}
<<Devuelve el valor máximo de una columna>> 
\begin{verbatim}
SELECT MIN(Ejemplares) FROM Ejemplo WHERE ID =1;
\end{verbatim}
<<Devuelve el valor mínimo de una columna>> 
\begin{verbatim}
SELECT COUNT(Ejemplares) FROM Ejemplo WHERE ID =1;
\end{verbatim}
<<Devuelve el numero de filas que coinciden con las condicione>> 
\begin{verbatim}
SELECT AVG(Ejemplares) FROM Ejemplo WHERE ID =1;
\end{verbatim}
<<Devuelve el valor promedio de una columna>> 
\begin{verbatim}
SELECT SUM(Ejemplares) FROM Ejemplo WHERE ID =1;
\end{verbatim}
<<Devuelve la suma de una columna numerica>> 
\begin{verbatim}
SELECT * FROM Ejemplo WHERE Nombre LIKE "a%";
\end{verbatim}
<<Muestra las personas cuyo nombre empieza por a>> 
\begin{verbatim}
SELECT * FROM Ejemplo WHERE Nombre LIKE "%a";
\end{verbatim}
<<Muestra las personas cuyo nombre acaba por a>> 
\begin{verbatim}
SELECT * FROM Ejemplo WHERE Nombre LIKE "%a%";
\end{verbatim}
<<Muestra las personas cuyo nombre contenga una a>> 
\begin{verbatim}
SELECT * FROM Ejemplo WHERE Nombre LIKE "_a%";
\end{verbatim}
<<Muestra las personas cuyo nombre contenga una a en segunda posicion>> 
\begin{verbatim}
SELECT * FROM Ejemplo WHERE nombre LIKE "a__%";
\end{verbatim}    
<<Muestra personas cuyo nombre empieza por a y tiene al menos 3 letras>>

\newpage

\begin{verbatim}
SELECT * FROM Ejemplo WHERE nombre LIKE "[abc]%";
\end{verbatim}
<<Muestra las personas cuyo nombre empieza por a, b o c>>
\begin{verbatim}
SELECT * FROM Ejemplo WHERE nombre LIKE "[!abc]%";
\end{verbatim}
<<Muestra las personas cuyo nombre no empieza por a, b o c>>
\begin{verbatim}
SELECT * FROM Ejemplo WHERE nombre NOT LIKE "[abc]%";
\end{verbatim}
<<Muestra las personas cuyo nombre no empieza por a, b o c>>
\begin{verbatim}
SELECT * FROM Ejemplo WHERE pais IN ('Espana', 'Cuba');
\end{verbatim}
<<Muestra las personas que estan en España o Cuba>>
\begin{verbatim}
SELECT * FROM Ejemplo WHERE pais NOT IN ('Espana', 'Cuba');
\end{verbatim}
<<Muestra las personas que no estan en España o Cuba>>
\begin{verbatim}
SELECT * FROM Ejemplo WHERE pais IN (SELECT pais FROM Ciudades);
\end{verbatim}
<<Muestra las personas que estan en los mismos paises que las ciudades>>
\begin{verbatim}
SELECT * FROM Ejemplo WHERE estatura BETWEEN 100 AND 200;
\end{verbatim}
<<Muestra las personas cuya estatura esta entre 100 y 200>>
\begin{verbatim}
SELECT * FROM Ejemplo WHERE estatura NOT BETWEEN 100 AND 200;
\end{verbatim}
<<Muestra las personas cuya estatura no esta entre 100 y 200>>
\begin{verbatim}
SELECT * FROM Ejemplo WHERE estatura BETWEEN 100 AND 200 
AND id NOT IN (1, 2);
\end{verbatim}
<<Muestra personas de estatura entre 100-200 y no las cuyo id es...>>
\begin{verbatim}
SELECT * FROM Ejemplo WHERE nombre BETWEEN "Alberto" 
AND "Manuel" ORDER BY nombre;
\end{verbatim}
<<Muestra las personas entre Alberto y Manuel ordenado por nombre>>
\begin{verbatim}
SELECT * FROM Ejemplo WHERE nombre NOT BETWEEN "Alberto" 
AND "Manuel" ORDER BY nombre;
\end{verbatim}
<<Muestra las personas que no estan entre Alberto y Manuel ordenado por nombre>>
\begin{verbatim}
SELECT * FROM Ejemplo WHERE FNac BETWEEN #01/07/20# AND #31/07/20#;
\end{verbatim}
<<Muestra las personas cuya fecha de nacimiento esta entre...>>
\begin{verbatim}
SELECT Idpersona AS Id, edad AS Ed FROM Ejemplo;   
\end{verbatim}
<<Crear alias para las columnas idpersona y edad>>

\newpage

\section*{Relaciones 0:N - 1:N (Uno a muchos)}
\addcontentsline{toc}{section}{Relaciones 0:N - 1:N}

\begin{verbatim}
ALTER TABLE Ejemplo ADD  (
Ordenador int(5) not null,
FOREIGN KEY(Ordenador) REFERENCES Ordenadores(IdPc)  );
\end{verbatim}
<<Creación del campo que relacciona los parámetros con clave secundaria>>
\begin{verbatim}
SELECT * 
FROM Ejemplo ej, Persona_Telefono pt, Telefonos t 
WHERE ej.Idpersonal = pt.Idpersonas AND t.Idtelefono 
= pt.Idtelefono;  
\end{verbatim}
<<.....>>
\begin{verbatim}
SELECT * 
FROM Ejemplo ej 
JOIN Persona_Telefono ejt ON p.Idpersonal = ejt.Idpersonas 
JOIN Telefonos t ON pt.Idtelefono = t.Idtelefono;  
\end{verbatim}
<<.....>>


\section*{Relaciones N:M (muchos a muchos)}
\addcontentsline{toc}{section}{Relaciones N:M}

*** PENDIENTE ***
\end{document}