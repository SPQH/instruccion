% INICIO DE DOCUMENTO
\documentclass[a4paper, 11pt, titlepage]{article}

% LIBRER?AS
\usepackage{hyperref}
\usepackage[utf8]{inputenc}
\usepackage{fancyhdr}
\usepackage{graphicx}
\usepackage{subcaption}
\usepackage{imakeidx}
\usepackage{makeidx}
\usepackage{mathtools}
\usepackage[spanish]{babel}
\usepackage{eurosym}
\usepackage{amssymb}

% VARIABLES
\newcommand{\nombre}{\emph{\textbf{Lingua::Romana::Perligata}}}

% CABECERAS
\title{\textbf{\nombre}\\ {\normalsize \textbf{Latin Perl Programming}}}
\author{Francisco Javier Balón Aguilar}
\date{2020}

% -------------------------------------------------------------------------------------------------------------------------------------

\begin{document} % INICIO DE DOCUMENTO

\maketitle % Creaci?n y dibujado del t?tulo como portada de documento
\renewcommand{\contentsname}{Índice} % Nombre dado al ?ndice
\tableofcontents % Genera la tabla de contenidos del ?ndice autom?ticamente
\newpage % Contin?a en una nueva p?gina el documento

\section{$\mathfrak{nomen}$}

Lingua::Romana::Perligata significa Perl en lengua romana (latín).

\section{$\mathfrak{editio}$}

This document describes version 0.601 of Lingua::Romana::Perligata released May 3, 2001.

\section{$\mathfrak{summarium}$}

\begin{verbatim}
    use Lingua::Romana::Perligata;
 
    adnota Illud Cribrum Eratothenis
    
    maximum tum val inquementum tum biguttam tum stadium egresso scribe.
    da meo maximo vestibulo perlegementum.
    
    maximum comementum tum novumversum egresso scribe.
    meis listis conscribementa II tum maximum da.
    dum damentum nexto listis decapitamentum fac
        sic
            lista sic hoc tum nextum recidementum cis vannementa listis da.
            dictum sic deinde cis tum biguttam tum stadium tum cum nextum
            comementum tum novumversum scribe egresso.
        cis
\end{verbatim}

\section{Bibliografía}

https://metacpan.org/pod/Lingua::Romana::Perligata

\end{document} % FIN DE DOCUMENTO
